%!TEX root = ../../main.tex

\chapter{Theoretische Grundlagen}
In den folgenden Kapiteln sollen die Theoretischen Grundlagen erläutert werden. Dafür wird zuerst betrachtet, wie das Lernen im Allgemeinen funktioniert und wie diese Erkenntnisse auf das Lernen von Eröffnungen angewendet werden können.
In dem darauffolgenden Kapitel wird betrachtet welche Erkenntnisse es konkret im Bereich der Schachpsychologie gibt. Dafür wird betrachtet, wie Schachspieler nach heutigem Wissensstand lernen und spielen.

\section{Allgemeines Lernen}
Über das Thema Lernen gibt es mehrere Theorien und Modelle. Aus diesen Theorien kann man auch teilweise Schlüsse ziehen, wie man das Erlernen von Schacheröffnungen gestalten kann.
Im Folgenden werden die Theorie des Behaviorismus, des Lernen am Modell und die strukturgenetische Lerntheorie kurz beschrieben.

\subsection{Behaviorismus}
Der Behaviorismus basiert auf tatsächlich beobachtbarem Verhalten. Er besagt, dass jeder Mensch durch seine Umwelt beeinflusst wird. Das bedeutet, dass man das Verhalten auch verändern kann durch Stimulation, primär durch positive Verstärkung und negative Verstärkung. Bei gutem Verhalten ist also Belohnung und Lob sinnvoll und bei schlechtem Verhalten können Strafen eingesetzt werden. Vor allem die positive Verstärkung sorgt dafür, dass eine Person extrinsische Motivation bekommt und auch gerne weiterlernt.\cite{kron_grundwissen_2024}

Bezogen auf eine Schachanwendung kann man zum Beispiel bei einem guten Zug positives Feedback geben zum Beispiel durch Lob oder durch positive Symbole, Farben und Animationen, wie zum Beispiel ein grüner Haken. Man könnte auch zum Beispiel Belohnungen einführen, wenn der Spieler mehrere aufeinanderfolgende Tage Übungen durchgeführt hat, oder sich verbessert hat.

\subsection{Lernen am Modell}
Eine weitere Theorie ist, dass Menschen Lernen am Modell. Das bedeutet, dass sie durch Beobachtung eines Modells neue Dinge erlernen können. Dieses Modell kann zum Beispiel eine andere Person sein.
Dieser Effekt tritt insbesondere dann auf, wenn die beobachtete Aktion in einem positiven Ergebnis resultiert. Dann ist die Wahrscheinlichkeit am höchsten, dass diese Aktion in einer ähnlichen Situation nachgeahmt wird.\cite{kron_grundwissen_2024}

In einer Schachanwendung kann man dieses Wissen so anwenden, dass man dem Spieler einen Zug vorführt und ihm dadurch zeigt,
%wie dieser zu einer vorteilhaften Position führen kann.
wie dieser einen Vorteil bringt.
In einer ähnlichen Position kann der Spieler sich dann eher an den Zug erinnern und ihn durchführen.

\subsection{Strukturgenetische Lerntheorie}
Die strukturgenetische Lerntheorie geht davon aus, dass ein Mensch am besten lernt, wenn er selbst verschiedene Handlungswege ausprobieren kann. Es geht darum, Personen in Herausforderungen zu versetzen und sie durch selbstständige Entdeckung etwas lernen zu lassen. Voraussetzung dafür ist, dass die Aktionen im Nachhinein reflektiert werden und dass die Person auch die notwendigen Kenntnisse und Mittel dafür hat. Dadurch kann sich dann Erfahrung bilden, denn \enquote{Erfahrung ist [...] immer reflektiertes oder durch Reflexion bestimmtes Tun.}\cite{kron_grundwissen_2024}

In einer Schachanwendung kann das durch die Implementierung von Rätseln geschehen. Man kann dem Spieler eine bestimmte Anfangsposition geben in welcher nur eine Auswahl an Zügen zu einer guten Position führen. Es wäre auch möglich ein Analysetool zur Verfügung zu stellen, mit welchem ein vergangenes Spiel reflektiert werden kann. So können gute und schlechte Züge identifiziert werden und der Spieler kann lernen, welche Züge gut funktioniert haben und welche nicht.

%TODO evtl Kapitel zu Auswendig lernen, wenn dazu Literatur auffindbar ist
