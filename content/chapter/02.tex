%!TEX root = ../../main.tex

\chapter{Theoretische Grundlagen}
In den folgenden Kapiteln sollen die Theoretischen Grundlagen erläutert werden. Dafür wird zuerst betrachtet, wie das Lernen im Allgemeinen funktioniert und wie diese Erkenntnisse auf das Lernen von Eröffnungen angewendet werden können.
In dem darauffolgenden Kapitel wird betrachtet welche Erkenntnisse es konkret im Bereich der Schachpsychologie gibt. Dafür wird betrachtet, wie Schachspieler nach heutigem Wissensstand lernen und spielen.

\section{Allgemeines Lernen}
Über das Thema Lernen gibt es mehrere Theorien und Modelle. Aus diesen Theorien kann man auch teilweise Schlüsse ziehen, wie man das Erlernen von Schacheröffnungen gestalten kann.
Im Folgenden werden die Theorie des Behaviorismus, des Lernen am Modell und die strukturgenetische Lerntheorie kurz beschrieben.
Danach wird auch die Funktionsweise unseres Gedächtnisses betrachtet.

\subsection{Behaviorismus}
Der Behaviorismus basiert auf tatsächlich beobachtbarem Verhalten. Er besagt, dass jeder Mensch durch seine Umwelt beeinflusst wird. Das bedeutet, dass man das Verhalten auch verändern kann durch Stimulation, primär durch positive Verstärkung und negative Verstärkung. Bei gutem Verhalten ist also Belohnung und Lob sinnvoll und bei schlechtem Verhalten können Strafen eingesetzt werden. Vor allem die positive Verstärkung sorgt dafür, dass eine Person extrinsische Motivation bekommt und auch gerne weiterlernt.\cite{kron_grundwissen_2024}

Bezogen auf eine Schachanwendung kann man bei einem guten Zug positives Feedback geben zum Beispiel durch Lob oder durch positive Symbole, Farben und Animationen. Man könnte auch Belohnungen einführen, wenn der Spieler mehrere aufeinanderfolgende Tage Übungen durchgeführt hat, oder sich verbessert.

\subsection{Lernen am Modell}
Eine weitere Theorie ist, dass Menschen lernen am Modell. Das bedeutet, dass sie durch Beobachtung eines Modells neue Dinge erlernen können. Dieses Modell kann zum Beispiel eine andere Person sein.
Dieser Effekt tritt insbesondere dann auf, wenn die beobachtete Aktion in einem positiven Ergebnis resultiert. Dann ist die Wahrscheinlichkeit am höchsten, dass diese Aktion in einer ähnlichen Situation nachgeahmt wird.\cite{kron_grundwissen_2024}

In einer Schachanwendung kann man dieses Wissen so anwenden, dass man dem Spieler einen Zug vorführt und ihm dadurch zeigt,
%wie dieser zu einer vorteilhaften Position führen kann.
wie dieser einen Vorteil bringt.
In einer ähnlichen Position kann der Spieler sich dann eher an den Zug erinnern und ihn durchführen.

\subsection{Strukturgenetische Lerntheorie}
Die strukturgenetische Lerntheorie geht davon aus, dass ein Mensch am besten lernt, wenn er selbst verschiedene Handlungswege ausprobieren kann. Es geht darum, Personen in Herausforderungen zu versetzen und sie durch selbstständige Entdeckung etwas lernen zu lassen. Voraussetzung dafür ist, dass die Aktionen im Nachhinein reflektiert werden und dass die Person auch die notwendigen Kenntnisse und Mittel dafür hat. Dadurch kann sich dann Erfahrung bilden, denn \enquote{Erfahrung ist [...] immer reflektiertes oder durch Reflexion bestimmtes Tun.}\cite{kron_grundwissen_2024}

In einer Schachanwendung kann das durch die Implementierung von Rätseln geschehen. Man kann dem Spieler eine bestimmte Anfangsposition geben in welcher nur eine Auswahl an Zügen zu einer guten Position führen. Es wäre auch möglich ein Analysetool zur Verfügung zu stellen, mit welchem ein vergangenes Spiel reflektiert werden kann. So können gute und schlechte Züge identifiziert werden und der Spieler kann lernen, welche Züge gut funktioniert haben und welche nicht.

\subsection{Funktionsweise des Gedächtnis}\label{gedächtnis}
Für das Erlernen von Schacheröffnungen ist es auch sinnvoll zu verstehen, wie das menschliche Gedächtnis funktioniert. Das Gedächtnis bildet die Grundlage des Lernens. Es wird als Netzwerk verstanden, das mit den Wahrnehmungsprozessen verbunden ist. Eine Besonderheit des menschlichen Gedächtnisses ist, dass sich Erinnerungen durch wiederholtes Erinnern verändern. Jedes mal, wenn man sich an etwas erinnert verändern sich die Erinnerungen ein wenig und es wird eventuell mehr Kontext hinzugefügt. Das bedeutet, dass eine Erinnerung, immer subjektiver wird und somit auch verfälscht werden kann. Außerdem kann es auch zum Vergessen kommen, indem vorhandene Erinnerungen überschrieben werden. Das Abspeichern von Informationen kann unterstützt werden durch verschiedene Lerntechniken. Dazu gehört zum Beispiel das Wiedergeben von Inhalten, Gruppieren von Inhalten oder Herausfiltern von Hauptideen. Gespeichert werden Inhalte auf drei unterschiedliche Arten. Das sensorische Gedächtnis ist für Inhalte zuständig, die jetzt im Moment wahrgenommen werden. Damit sind äußere Reize und innere Zustände gemeint. Das Kurzzeitgedächtnis verarbeitet aktuelle Informationen zu Wissen. Seine Kapazität ist allerdings stark begrenzt. Deshalb können Inhalte nicht lange im Kurzzeitgedächtnis behalten werden, sie durch neue Inhalte ersetzt und verdrängt. Im Langzeitgedächtnis werden Inhalte des Kurzzeitgedächtnisses übernommen. Dieser Prozess kann durch Lerntechniken unterstützt werden. Das Langzeitgedächtnis hat eine sehr große Kapazität und behält Informationen lange um später darauf zurückgreifen zu können. Hier wird wiederum zwischen dem deklarativen und dem prozeduralen Gedächtnis unterschieden. Das deklarative Gedächtnis ist zuständig für bewusst abrufbare Inhalte, wie persönliche Erlebnisse, Fakten und allgemein Inhalte die mit Worten kommuniziert werden können. Das prozedurale Gedächtnis enthält auch Wissen, das nicht unbedingt bewusst wahrgenommen wird. Dort sind auch zum Beispiel motorische Kenntnisse gespeichert.
\cite{kron_grundwissen_2024}

Aus diesen Erkenntnissen kann man schlussfolgern, dass es besonders wichtig ist Inhalte zu wiederholen, damit sich Gelerntes nicht verfälscht und es nicht vergessen wird. Im Schach sollte man um Eröffnungen zu erlernen diese also häufig betrachten. Durch das eigene Spielen und Erinnern kann das Übergehen dieser Eröffnungen in das Langzeitgedächtnis unterstützt werden. Bestimmte Variationen, welche seltener vorkommen sollten auch wiederholt werden, um ein Vergessen, Überschreiben oder Verfälschen zu verhindern.

\section{Schachpsychologie}% Psychologie des Eröffnungen Lernens
In der Schachpsychologie wurde von unterschiedlichen Personen bereits einige gemeinsame Beobachtungen gemacht. Gobet und Jansen haben in \cite{gobet_training_2006} die folgende Kernaussagen gesammelt:

\begin{enumerate}
    \item Ein Schachspieler hat einen hocheffizienten Überblick. Er erkennt zentrale Elemente einer Position sehr schnell.
    \item Schachspieler können sich Schachpositionen und Spiele außergewöhnlich gut merken. Diese Fähigkeit ist außerhalb von Schach nicht erkennbar.
    \item Ihr Wissen besteht aus mehreren Ebenen, im speziellen die niedrige Ebene, welche aus Muster von Figuren besteht und eine hohe konzeptionelle Ebene, welche sich mit Plänen und Bewertungen von Positionen befasst.
    \item Die Spieler suchen sehr selektiv nach guten Zügen. Es werden nur ganz bestimmte Pfade durchdacht und andere sehr schnell verworfen.
    \item Es gibt keinen Unterschied in dem Suchalgorithmus eines Expertenspielers und eines Großmeisters.
    \item Meister verlieren in simultanen Spielen relativ wenig ihrer Fähigkeiten.
\end{enumerate}

\subsection{Die Template Theorie}
Aufschluss über diese Beobachtungen soll die Template Theorie geben, welche von Gobet und Simon in \cite{gobet_templates_1996} beschrieben wird. Die Template Theorie berücksichtigt, dass das kognitive System der Menschen aus drei Hauptbestandteilen besteht, wie in \autoref{gedächtnis} beschrieben.
Das sensoriesche Gedächtnis wird hier räumlich-zeitlicher Speicher genannt und es wird auch das stark begrenzte Kurzzeitgedächtnis und das Langzeitgedächtnis näher betrachtet.
Bei dem Langzeitgedächtnis wird wieder zwischen deklarativem und prozeduralem Wissen unterschieden.
\cite{gobet_training_2006}

Bei dem deklarativen Wissen in Bezug auf Schach wird vermutet, dass Chunks eine große Rolle spielen. Chunks sind kleine Muster, also eine Anordnung von einigen Spielfiguren, die öfter bei einem Schachspiel vorkommen. Es wird geschätzt, dass das Kurzzeitgedächtnis Platz für ungefähr sieben Chunks hat. Ein großer Unterschied zwischen geübten und weniger geübten Schachspielern ist also, wie viele Chunks sie in ihrem Langzeitgedächtnis haben. Je besser ein Schachspieler ist, desto mehr und desto größere Chunks befinden sich in seinem Langzeitgedächtnis. Es wird vermutet, dass ein geübter Spieler ungefähr 50.000 Chunks in seinem Langzeitgedächtnis hat. Wenn ein Muster erkannt wird, muss im Kurzzeitgedächtnis nur ein Zeiger auf den Chunk im Langzeitgedächtnis behalten werden. Auf diese Weise können geübte Schachspieler sich sehr effizient Schachpositionen merken. Das erklärt auch, warum sich diese Fähigkeit nicht auf Bereiche außerhalb des Schachs auswirkt. Selbst zufällig angeordnete Positionen können sich gute Spieler schlecht merken. Erweitert wird diese Theorie durch Templates, also Vorlagen. Templates sind spezielle Chunks, welche Platzhalter beinhalten für bestimmte Figuren. So können durch Templates noch mehr Positionen abgedeckt werden, als es durch einfache Chunks möglich ist. Viele Chunks und Templates reduzieren die Notwendigkeit voraus zu schauen. Wenn bekannten Chunks begegnet wird, ist auch klar, wie gut diese Position zu bewerten ist und welche Züge in Frage kommen.\cite{gobet_templates_1996}

Dieses Wissen, was als nächstes getan werden kann, wird von Gobet und Jansen unter dem prozedurales Wissen eingeordnet. Es wird durch sogenannte Produktionen abgespeichert. Produktionen sind Wissenseinheiten, welche aus Bedingung und Aktion bestehen. Ein Beispiel wäre, \enquote{Wenn es eine Linie ohne Figuren gibt und du einen Turm besitzt, dann platziere den Turm auf dieser Reihe.} Durch diese Produktionen können Spieler schnell Entscheidungen treffen. Dieser Mechanismus kann bewusst oder unbewusst stattfinden und wird oft als Intuition verstanden.\cite{gobet_training_2006}

\subsection{Erlernen von Eröffnungen}

