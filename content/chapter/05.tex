%!TEX root = ../../main.tex

\chapter{Fazit}
Im Rahmen dieser Arbeit wurde untersucht, wie Menschen im Allgemeinen lernen und wie Schachspieler lernen. Dabei zeigte sich, dass positive Verstärkung, gute Vorbilder, interaktives Lernen und Wiederholung besonders wichtig sind.
Bei einem Thema wie Schacheröffnungen liegt der Fokus darauf, die Inhalte oft zu wiederholen, um zu verhindern, dass sich die Erinnerungen verfälschen oder in Vergessenheit geraten. Im Optimalfall wiederholt man die Inhalte in exponentiell größer werdenden Intervallen, mithife von Spaced Repetition. Das erwies sich bereits beim Lernen von Fremdsprachen als effizient.

In \autoref{cp:schachpsychologie} wurde erläutert, dass Chunks und Templates, also kleine Muster von Spielfiguren, eine zentrale Rolle spielen beim Erlernen von Schach. Je mehr Chunks ein Spieler im Langzeitgedächtnis besitzt, umso besser ist er auf verschiedene Positionen vorbereitet. Die Forscher sind sich auch einig, dass Computer beim Schach lernen nützlich sein können. Computer spielen mittlerweile auch besser als es Menschen können, durch Algorithmen wie Alpha-Beta Pruning oder durch Reinforcement Learning trainierte Modelle.

In \autoref{cp:marktanalyse} wurde herausgefunden, dass es unterschiedliche Medien gibt, die einen Schachspieler beim Erlernen von Eröffnungen unterstützen. Diese sind aber oft nicht kostenlos oder bieten wenig Personalisierung. Daher wurde ein neues Programm erstellt, um diese Lücke zu schließen. Im Rahmen der Recherche wurden dafür Datensätze und Schachengines gefunden, die für die Erstellung einer solchen Schachlernanwendung hilfreich sind.

\section{Implementierung}
Die Architektur der Implementierung wurde modular gestaltet und durch Verwendung von Patterns wie Dependency Injection und das Repository Pattern, sind Erweiterungen unkompliziert möglich. Die Dependency Injection vereinfachte vor allem die Erstellung von Unit Tests, was die Wartbarkeit des Programms verbessert. Weil das Frontend und das Backend durch eine klar definierte REST-Schnittstelle getrennt ist, war eine parallele Entwicklung beider Bestandteile möglich. Die gefundenen Eröffnungssammlungen ermöglichten es, sehr viele Eröffnungen mit vergleichsweise kleinem Aufwand einzubinden. Durch die Implementierung von Statistiken ist es möglich, die Benutzererfahrung zu personalisieren und seinen Fortschritt zu verfolgen.

Das Programm erfüllt die meisten, der im Voraus definierten Anforderungen. Mit ihm kann man Eröffnungen betrachten und üben. Es kann als zentrales Nachschlagewerk genutzt werden, um sich Eröffnungen nochmal ins Gedächtnis zu rufen. Außerdem wird dem Nutzer vorgeschlagen, welche Eröffnung er als nächstes wiederholen soll, aufgrund eines berechneten Expertisenwertes. Aus zeitlichen Gründen war es nicht mehr möglich die Übungsmodus Erweiterung umzusetzen. Das Programm kann dennoch gut als Ergänzung zu bestehenden Lernmedien verwendet werden. So kann man zum Beispiel einen Trainer oder ein Buch verwenden, um die Kernideen einer Eröffnung zu verstehen und das Programm, um die Eröffnung zu wiederholen und sein Wissen zu festigen.

\section{Ausblick}
Die Implementierung kann noch in einigen Aspekten verbessert werden. Ein Punkt wurde bereits bei den Statistiken genannt. Die Vergessenskurve und damit auch der Expertisenwert, der ausschlaggebend für die Empfehlungen ist, wird aufgrund von geratenen Parametern berechnet. Es wäre möglich, durch die Statistiken der Anwendung Daten zu sammeln, um die Parameter durch Regression zu bestimmen. Das würde die Berechnung des Expertisenwertes deutlich verbessern und die Vorschläge, welche Eröffnung als nächstes wiederholt werden soll, wären akkurater.
Durch positives Feedback könnte man Nutzer dazu motivieren häufiger zu trainieren. Dafür können positive Farben und Animationen verwendet werden, oder man orientiert sich an den Belohnungen von Chessable und bindet auch ein Punktesystem oder Belohnungen, wie zum Beispiel Abzeichen, ein.
Eine nützliche Erweiterung wären auch erklärende Infotexte, um das tiefere Verständnis von Eröffnungen verstärken.
Ein Nachteil der Anwendung ist, dass sich in dem verwendeten Datensatz sehr viele Eröffnungen befinden. Das kann vor allem für Anfänger überfordernd sein. Empfehlungen für ungeübte Eröffnungen könnten hier Abhilfe verschaffen, denn die Vorschläge in der aktuellen Implementierung begrenzen sich auf Eröffnungen, die bereits geübt wurden. Es wäre auch denkbar eine Eröffnungsdatenbank einzubinden, um Nutzern zu zeigen, wie beliebt verschiedene Eröffnungen sind.
Zu guter Letzt kann man durch Einbindung einer Schachengine die Übungsmodus Erweiterung ergänzen. Dann könnten Nutzer nach einer Eröffnung gegen einen Computergegner weiter spielen. Das verbessert das Verständnis zwischen Eröffnungen und Mittelspiel.
Die Anwendung könnte auch erweitert werden, indem Übungen für das Mittel- und Endspiel angeboten werden. Das kann zum Beispiel durch Puzzles umgesetzt werden.
