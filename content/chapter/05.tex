%!TEX root = ../../main.tex

\chapter{Fazit}
Im Rahmen dieser Arbeit wurde untersucht, wie Menschen im Allgemeinen lernen und wie Schachspieler lernen. Dabei zeigte sich, dass es für ein Thema wie Schacheröffnungen wichtig ist die Inhalte oft zu wiederholen, um zu verhindern, dass sich die Erinnerungen verfälschen oder vergessen. Im Optimalfall wiederholt man die Inhalte in exponentiell größer werdenden Intervallen, in Form von Spaced Repetition. In der Marktanalyse wurde herausgefunden, dass es einige Medien gibt, die einen beim Erlernen von Eröffnungen unterstützen. Diese sind aber oft nicht kostenlos oder personalisiert. Daher wurde ein neues Programm erstellt, dass diese Lücke schließen soll. Im Rahmen der Recherche wurden auch Datensätze und Schachengines gefunden, die für die Erstellung einer solchen Schachlernanwendung hilfreich sind.

\section{Implementierung}
In dem implementierten Programm wurden die meisten, der im Voraus definierten Anforderungen, erfüllt. Mit ihm kann man Eröffnungen betrachten und auch üben. Es kann als zentrales Nachschlagewerk genutzt werden, um sich Eröffnungen nochmal ins Gedächtnis zu rufen. Außerdem wird dem Nutzer vorgeschlagen, welche Eröffnung er als nächstes wiederholen soll, aufgrund eines berechneten Expertisenwertes. Durch die Einbindung eines öffentlichen Datensatzes stehen sehr viele Eröffnungen zur Auswahl. Aus zeitlichen Gründen war es nicht mehr möglich die Übungsmodus Erweiterung umzusetzen. Das Programm kann gut als Ergänzung zu bestehenden Lernmedien verwendet werden. So kann man zum Beispiel einen Trainer oder ein Buch verwenden, um die Kernideen einer Eröffnung zu verstehen und das Programm verwenden, um die Eröffnung zu wiederholen und sein Wissen zu festigen.

\section{Ausblick}
Das entwickelte Programm kann noch in einigen Aspekten verbessert werden. Ein Punkt wurde bereits bei den Statistiken genannt. Die Vergessenskurve und damit auch der Expertisenwert, der ausschlaggebend für die Empfehlungen ist, wird aufgrund von geratenen Parametern berechnet. Es wäre möglich durch die Statistiken der Anwendung Daten zu sammeln, um die Parameter durch Regression zu bestimmen. Das würde die Berechnung des Expertisenwertes deutlich verbessern und die Vorschläge, welche Eröffnung als nächstes wiederholt werden soll, wären akkurater.
Durch positives Feedback könnte man Nutzer dazu motivieren häufiger zu trainieren. Dafür kann man positive Farben und Animationen verwenden, oder orientiert sich an den Belohnungen von Chessable und bindet auch ein Punktesystem oder Belohnungen ein.
Eine nützliche Erweiterung wären auch erklärende Infotexte, um das tiefere Verständnis von Eröffnungen verstärken.
Ein weiteres Problem ist, dass es in der Anwendung sehr viele Eröffnungen gibt. Das kann vor allem für Anfänger überfordernd sein. Vorteilhaft wäre eine Empfehlung auch für ungeübte Eröffnungen, denn die Vorschläge in der aktuellen Implementierung begrenzen sich auf Eröffnungen, die bereits geübt wurden. Es wäre auch denkbar eine Eröffnungsdatenbank einzubinden, um Nutzern zu zeigen, wie beliebt die Eröffnungen sind.
Zu guter letzt kann man auch durch Einbindung einer Schachengine die Übungsmodus Erweiterung ergänzen. Dann könnten Nutzer von einer Eröffnung aus gegen einen Computergegner weiter spielen. Das verbessert das Verständnis zwischen Eröffnungen und Mittelspiel.
Die Anwendung könnte auch erweitert werden, indem Übungen für das Mittel- und Endspiel angeboten werden. Das kann zum Beispiel durch Puzzles umgesetzt werden.
