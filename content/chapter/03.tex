%!TEX root = ../../main.tex

\chapter{Marktanalyse}
Wie bereits erwähnt gibt es viele Programme, die eine Person beim Erlernen von Schacheröffnungen unterstützen kann. In diesem Kapitel wird zuerst betrachtet, welche Lernplattformen es gibt und ob es in dem aktuellen Angebot Defizite gibt, die durch unser Projekt ausgeglichen werden können. Anschließend findet ein Vergleich von Schachengines und von Datenbanken statt unter der Betrachtung, welche am besten in dieses Projekt eingebunden werden können.


\section{Lernplattformen}
Wenn man bereit ist Geld auszugeben, kann man sich zwischen einigen unterschiedlichen online Lernplattformen entscheiden. Bei kostenlosen Plattformen ist die Auswahl kleiner. Diese Plattformen haben einige Gemeinsamkeiten, aber auch zum Teil größere Unterschiede. In diesem Abschnitt werden die kostenpflichtigen Plattformen Chess.com und Chessable und die kostenlose Plattform Lichess verglichen.

\subsection{Chess.com}
Chess.com ist mit seinen über 100 Millionen Mitgliedern eine der größten Schachplattformen im Internet. \cite{chesscom_chesscom_2022} Neben der Möglichkeit gegen andere Mitspieler oder Computergegner zu spielen gibt es auch interaktive Lektionen um Schach zu erlernen. Einige darunter sind auch um Eröffnungen lernen. Um Zugriff auf die Lektionen zu bekommen muss man allerdings ein kostenpflichtiges Abonnement abschließen für aktuell 4,17€ pro Monat. Die Lektionen sind immer ähnlich aufgebaut. Am Anfang kann man ein erklärendes Video anschauen. Anschließend werden bestimmte Teile der Eröffnung abgefragt. Dadurch, dass in einem Video mehrere Varianten einer Eröffnung erklärt werden, können diese relativ lang werden. Teilweise sind sie über 30 Minuten und das kann es erschweren alle Informationen auf einmal aufzunehmen. Außerdem kann es auch nachteilhaft sein, dass nicht die gesamte Eröffnung abgefragt wird, sondern nur bestimmte Ausschnitte. Auf der Seite gibt es auch die Möglichkeit seine vergangenen Spiele zu analysieren. Ein Teil der Analyse ist auch kostenfrei zur Verfügung. Insgesamt kann man sagen, dass die Interaktivität der Kurse etwas begrenzt ist.

\subsection{Lichess}
Lichess ist eine vollkommen kostenlose Plattform. Auf der Seite gibt es offizielle interaktive Kurse um Grundlagen und auch fortgeschrittenere Themen zu lernen. Eröffnungen kann man über die sogenannten Studien lernen. Das sind Kurse, die von anderen Benutzern erstellt wurden. Dadurch variieren sie auch in der Qualität und es ist nicht für jede Eröffnung ein guter Kurs vorhanden. Auch bei Lichess hat man die Möglichkeit seine vergangenen Partien zu analysieren.

\subsection{Chessable}
Chessable ist eine Seite, die betont, dass sie durch wissenschaftliche Erkenntnisse das Lernen so einfach wie möglich gestalten wollen. \cite{prof_barry_hymer_science_nodate} Im Zentrum steht dabei Spaced-Repetition. Dabei geht es darum, dass ein Mensch sich Dinge besser merken kann, wenn sie in einem größer werdenden Abstand wiederholt werden, abhängig davon, ob man sich richtig erinnert hat oder nicht. Außerdem sei die Seite, wie ein Spiel aufgebaut mit Belohnungen wie Punkten und Abzeichen. Dadurch soll Dopamin ausgeschüttet werden, was den Lerneffekt erhöht. Ohne Geld auszugeben hat man bei Chessable allerdings nur eine kleine Auswahl an Kursen und Funktionen. Es gibt zum einen die Möglichkeit ein Abonnement für 11.99\$ abzuschließen und so Zugriff auf viele Funktionen und Kurse zu bekommen, zum anderen kann man auch einzelne Kurse kaufen. Diese Kurse können teuer werden, einige sind über 100\$. Sie sind meistens so gestaltet, dass ein Zug vorgeführt wird und erklärt wird. Danach spielt man selbst den Zug nach. Am Ende einer Variante wird die gesamte Variante abgefragt. Es wird also gelernt durch die Imitation des Gesehenen und durch Wiederholung. Man erkennt bei Chessable, dass sie einen Fokus darauf gesetzt haben wissenschaftliche Erkenntnisse mit einzubeziehen um das Lernen so einfach wie möglich zu gestalten. Die Plattform an sich ist allerdings verhältnismäßig teuer, vor allem wenn man neben dem Abonnement noch zusätzliche Kurse kaufen will.

\subsection{Ergebnis}
Die betrachteten Plattformen variieren in ihrer Personalisierbarkeit. Lichess ist weniger personalisiert, hat ein begrenztes Angebot an Kursen, ist aber im Gegensatz zu Chess.com und Chessable kostenfrei. Chess.com ist eine etablierte Marke und bietet einige interaktive Kurse, die das Lernen erleichtern. Personalisierung ist allerdings begrenzt. Chessable hat am meisten Personalisierung und auch am meisten Anreize das Lernen attraktiv zu machen durch Punkte, Abzeichen und Spaced-Repetition. Es ist allerdings auch die Seite mit den höchsten Preisen. Unser Programm soll auch Vorschläge machen, welche Kurse als nächstes gelernt werden sollen. Zusätzlich so es  möglich sein über Statistiken zu betrachten, welche Eröffnungen man bereits gut kann und in welchen man sich verbessern kann. Erklärungen oder Videos zu den einzelnen Eröffnungsvariationen liegen außerhalb des Umfangs.


\section{Schachengines}
Mittlerweile existieren sehr viele Schachengines, die sich leicht in ihren Algorithmen unterscheiden. Einen Startpunkt für den Vergleich der Engines bietet der \ac{TCEC}. In diesem Turnier treten unterschiedliche Schachengines gegeneinander an in mehreren Spielen. Der Sieger einer Saison wird  \ac{TCEC} Champion. In der Vergangenheit waren die Engines Stockfish, LCZero, Komodo und Houdini Champions. \cite{tcec_chessdom_tcec-chess_2025}

Ein Team hat auch Statistiken angelegt über die meisten Schachengines. Das Ergebnis ist die \acf{CCRL}. \cite{ccrl_team_ccrl_2025} \autoref{tab:engines} zeigt eine Übersicht von vier Engines, die genauer betrachtet werden. Es wird ihre Elo-Bewertung laut \ac{CCRL} und die Anzahl ihrer \ac{TCEC}-Siege aufgelistet.

\begin{table}[h]
    \centering
    \begin{tabular}{|c|l|c|c|c|}
        \hline
        Engine & Algorithmus & Elo & Siege & kostenfrei \\
        \hline
        Stockfish & Alpha-Beta + \acs{NNUE} & 3642 & 13 & ja \\
        \hline
        Komodo & \acs{MCTS} + \acs{NNUE} & 3626 & 4 & nein \\
        \hline
        LCZero & \acs{MCTS} & 3443 & 2 & ja \\
        \hline
        AlphaZero & \acs{MCTS} & - & - & nein \\
        \hline
    \end{tabular}
    \caption{Schachengines}
    \label{tab:engines}
\end{table}

\subsection{AlphaZero}
AlphaZero war ein Forschungsprojekt, dass in einem vorherigen Kapitel näher beschrieben wurde. Die Engine wurde hauptsächlich für Forschungszwecke entwickelt und ist nicht frei zur Verwendung verfügbar. Es haben sich allerdings einige andere Programmierer an den Forschungsergebnissen der Engine orientiert und ähnliche Engines entwickelt.

\subsection{LCZero}
LCZero ist ausgeschrieben Leela Chess Zero und wird auch mit dem Namen lc0 bezeichnet. Diese Engine verwendet den gleichen Ansatz, wie AlphaZero mit ein paar Anpassungen. Sie arbeitet also mit einem Neuralen Netzwerk um die besten Züge herauszufinden. Im Gegensatz zu AlphaZero kann jeder LCZero mit einem vortrainierten Neuralen Netz herunterladen und über \ac{UCI} verwenden.
\cite{noauthor_neural_2020}

\subsection{Komodo}
Komodo gewann vor ein paar Jahren einige \ac{TCEC}. Komodo wird auch von Chess.com verwendet für Computergegner mit unterschiedlicher Stärke. Seit Version 12 ist auch eine \ac{MCTS}-Engine integriert. Die neuesten Versionen heißen Komodo Dragon und verwenden auch ein \ac{NNUE}. Die Engine wird auch über \ac{UCI} angesteuert. Die neuesten Versionen sind allerdings kostenpflichtig.
\cite{wikipedia_foundation_inc_komodo_2024}

\subsection{Stockfish}
Stockfish verwendet Alpha-Beta-Pruning um die besten Züge herauszufinden. Früher wurden die Schachpositionen durch eine handgeschriebene Evaluationsfunktion bewertet. Mittlerweile wurde diese Funktion durch ein \ac{NNUE} ersetzt, was Stockfish deutlich verbesserte. Die Engine wird oft als die stärkste Schachengine angesehen. Sie hat auch die meisten \ac{TCEC} gewonnen. Die letzten sechs Saisons blieb sie ungeschlagen. Stockfish ist auch kostenlos verfügbar und kann über \ac{UCI} verwendet werden.
\cite{wikipedia_foundation_inc_stockfish_2025}

\subsection{Auswahl}
Für dieses Projekt wird die Schachengine Stockfish verwendet. Sie ist die aktuell stärkste Schachengine und auch kostenlos. Sie kann von der Stärke auch angepasst werden und über \ac{UCI} angesteuert werden. Das macht sie geeignet um sie in ein anderes Programm einzubinden.

\section{Eröffnungssammlungen}
In diesem Kapitel wird beschrieben, welche Eröffnungssammlungen existieren und es wird bewertet, wie gut sie für eine Lernanwendung für Eröffnungen verwendet werden können.

\subsection{Encyclopedia of Chess Openings}
Die Encyclopedia of Chess Openings ist eine der bekanntesten Sammlungen von Schacheröffnungen. Sie wurde erstellt, um den aktuellen Stand der Eröffnungstheorie festzuhalten.
Die Editoren, von denen die meisten Großmeister sind, wählten aus den Partien von Schachmeistern die Züge aus, die sie für am bedeutendsten hielten. Die Eröffnungen wurden in die fünf Kategorien A bis E aufgeteilt, für die jeweils ein Buch veröffentlicht wurde. Die Kategorien sind wiederum unterteilt in hundert nummerierte Unterkategorien. Die Kategorien und ihre Nummern werden auch ECO Codes genannt.
\cite{wikipedia_foundation_inc_encyclopaedia_2024}
\todo{mehrdeutigkeit von ECO Codes}

Mit dieser Sammlung wären die wichtigsten Eröffnungen abgedeckt. Allerdings ist die Encyclopedia of Chess Openings in erster Linie eine Buchreihe, in der die Eröffnungen in Tabellenform dargestellt wird mit textuellen Beschreibungen in natürlicher Sprache. Um die Eröffnungen in einem Programm zu verwenden, müssen sie allerdings in maschinenlesbarer Form vorliegen.

\subsection{Eröffnungsdatenbanken}
Eröffnungsdatenbanken werden auch Opening Explorer genannt. Sie besitzen Datensätze von vielen vergangenen Schachspartien. In der Benutzeroberfläche kann man betrachten, welche Züge wie häufig in diesem Datensatz gespielt wurden und wie oft schwarz oder weiß nach dieser Kombination gewonnen hat. 
% ChessBase ist das bekannteste Beispiel für eine solche Datenbank. Die meisten Funktionen von ChessBase verlangen allerdings ein Abonnement. Andere Seiten, wie Chess.com und Lichess bieten einen Teil der Funktionen auch kostenlos an.

Eröffnungsdatenbanken sind ein nützliches Werkzeug, um sein Wissen über bekannte Eröffnungen zu vertiefen und zu erfahren, welche Varianten häufig gespielt werden. Sie sind allerdings nicht dafür gedacht, neue Schacheröffnungen zu erlernen. Sie können auch nicht in ein anderes Programm eingebunden werden, da sie meist nur eine grafische Schnittstelle besitzen und keine \ac{API}.

\subsection{Eröffnungsbücher für Computer}
Die Datensätze hinter den Eröffnungsdatenbanken werden als Eröffnungsbücher bezeichnet. Sie werden auch oft von Schachengines genutzt, um in den Anfangszügen Rechenleistung und Zeit zu sparen. Es gibt sie in Binär- und Textformaten. Die Binärformate sind häufig nicht vollständig dokumentiert, mit der Ausnahme von Polyglott. Die Textformate sind meist in \ac{PGN}.
Bei dieser Notation werden die Züge in der verkürzten algebraischen Notation festgehalten. Die algebraische Notation ist die, die auch in Eröffnungsbüchern für Menschen genutzt werden.
\cite{wikipedia_foundation_inc_chess_2025}

Im Internet kann man auch einige kostenlose Polyglott- oder PGN-Eröffnungsbücher finden. Diese haben nicht unbedingt den selben Umfang, wie in kostenpflichtigen Datenbanken, können aber dennoch verwendet werden um selber eine Eröffnungsdatenbank zu implementieren. Für ein Programm zum Erlernen von Schacheröffnungen ist es allerdings nicht notwendig mehrere Millionen von vergangenen Partien zu speichern und zu analysieren. Eine Liste mit den häufigsten Eröffnungen und ihren Varianten wäre bereits ausreichen.
% Außerdem ist auch nicht eindeutig, wo eine Eröffnung endet, da sich in den Dateien immer vollständige Spiele befinden.

\subsection{Eröffnungslisten}
Es existieren auch einige Eröffnungslisten, die keine vollständigen Partien enthalten sondern nur die Eröffnungen. \cite{omur_yanikoglu_ecojson_2025} ist zum Beispiel eine Sammlung an Eröffnungen im JSON-Format. Sie enthält Eröffnungen aus der Encyclopedia of Chess Openings, der SCID Sammlung und von Wikipedia. Als Basis dient wiederum die Sammlung \cite{lichessorg_chess-openings_2025} von Lichess. Diese enthält die Eröffnungen der Encyclopedia of Chess Openings in TSV-Dateien, die den ECO-Code, den englischen Namen, die Zugfolge in \ac{PGN} und \ac{UCI}-Notation und die Endposition enthält.

Die Sammlung von Lichess ist in einem Format, das gut für eine Lernanwendung geeignet ist. Sie ist mit mehreren tausend Eröffnungen allerdings sehr umfangreich und sollte noch gefiltert werden. Es gibt in der Datenstruktur auch keine direkte Gruppierung für die Eröffnungsfamilien. Die Familien können aber durch den englischen Namen erkannt werden. Er besitzt die Struktur \lstinline{Familie: Variation, Untervariation, ...}.
