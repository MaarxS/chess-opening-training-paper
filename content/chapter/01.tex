%!TEX root = ../../main.tex

\chapter{Einleitung}

Schach ist ein weithin bekanntes und beliebtes Spiel. In den letzten Jahren gab es einen großen Anstieg an Beliebtheit verursacht durch die Pandemie und online Schach-Plattformen, wie Chess.com. Auch die Serie Queens Gambit und die vergrößerte Präsenz von Schach in sozialen Medien haben stark dazu beigetragen. Im Jahr 2022 schrieb Chess.com\: \enquote{In [den letzten] 18 Monaten haben wir mehr neue Mitglieder gewonnen, als in den 13 Jahren zuvor zusammen.} \cite{chesscom_chesscom_2022} Aus diesem Grund gibt es auch immer mehr Personen, die besser werden wollen in Schach und Strategien lernen wollen. Ein wichtiger Teil der Strategien sind die Eröffnungen. Die ersten Züge eines Spiels entscheiden oft in welche Richtung sich das Spiel entwickelt. Es gibt viele Bücher über Eröffnungen und weitere Medien wie Videos und online Kurse. Diese Medien sind allerdings nicht sehr interaktiv und das Lernen wird zu einem auswendig lernen. Das ist für viele nicht so einfach und die Eröffnungen werden schnell wieder vergessen. Durch ein Computeranwendung kann der Prozess interaktiver gestaltet werden. Chess.com bietet auch einen interaktives Training für Eröffnungen an, allerdings wird dafür ein monatliches Abonnement benötigt und die Trainings sind nur begrenzt interaktiv. Ziel dieser Arbeit ist es eine Schach-Trainingsanwendung zu entwickeln, welche es Spielern ermöglicht Eröffnungen interaktiv zu erlernen. 

