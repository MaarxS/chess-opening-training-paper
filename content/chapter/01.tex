%!TEX root = ../../main.tex

\chapter{Einleitung}

Schach ist ein weit verbreitetes und beliebtes Spiel. In den letzten Jahren gab es einen großen Anstieg an Beliebtheit, verursacht durch die Pandemie und Online"=Schachplattformen, wie Chess.com. Auch die Serie Queens Gambit und die vergrößerte Präsenz von Schach in sozialen Medien haben stark dazu beigetragen. Im Jahr 2022 schrieb Chess.com\: \enquote{In [den letzten] 18 Monaten haben wir mehr neue Mitglieder gewonnen, als in den 13 Jahren zuvor zusammen.} \cite{chesscom_chesscom_2022}
Aus diesem Grund gibt es auch immer mehr Personen, die ihr Schachspiel verbessern und Strategien lernen wollen.

\section{Motivation}
Ein wichtiger Teil der Strategien sind die Eröffnungen. Die ersten Züge eines Spiels entscheiden oft in welche Richtung sich das Spiel entwickelt. Durch eine gut geplante Eröffnung kann ein Spieler frühzeitig die Kontrolle über das Zentrum gewinnen, die Figuren optimal platzieren und Schwachstellen minimieren. Wer hingegen nicht vertraut mit Eröffnungen ist, wird schnell zum Opfer von Fallen und taktischen Manövern. Im schlimmsten Fall verliert man ein Spiel, noch bevor es richtig begonnen hat. Mit einem fundierten Wissen über Eröffnungen können auch Schwachstellen in typischen Stellungen schneller erkannt und ausgenutzt werden. Auf diese Weise kann man sich frühzeitig einen Vorteil verschaffen und das Spiel zu seinen Gunsten gestalten. Das Erlernen von Schacheröffnungen gestaltet sich häufig anspruchsvoll, da sie nicht nur komplex sind, sondern auch schnell wieder in Vergessenheit geraten können. Daher ist es essenziell, Eröffnungen auf eine systematische und regelmäßige Weise zu trainieren.

Zum Erlernen von Schacheröffnungen existieren viele Bücher Eröffnungen und weitere Medien, wie Videos und Online-Kurse. Diese Materialien eignen sich gut um ein Verständnis von Eröffnungen aufzubauen und die zentralen Ideen und Ziele dahinter zu verstehen.
Sie sind jedoch oft wenig interaktiv, wodurch das Lernen häufig auf reines Auswendiglernen reduziert wird. Für viele Lernende ist dadurch das Lernen erschwert  und Eröffnungen werden schnell wieder vergessen. Durch eine Computeranwendung kann der Prozess interaktiver gestaltet werden. Chess.com bietet ein interaktives Training für Eröffnungen an. Diese Interaktivität beschränkt sich allerdings darauf, dass der Spieler vorgeschriebene Züge nachspielen soll. Ein weiterer Nachteil ist, dass ein monatliches Abonnement benötigt um Zugriff auf das Training zu bekommen. Für Spieler, die nicht monatlich Geld ausgeben wollen gibt es kaum Alternativen.
Es existiert zu diesem Zeitpunkt kein Programm, das ein individuell zugeschnittenes und personalisiertes Lernen von Schacheröffnungen ermöglicht, bei dem spezifische Bedürfnisse, Stärken und Schwächen der Lernenden gezielt berücksichtigt werden.

\section{Ziel der Arbeit}
Ziel dieser Arbeit ist es eine Schach-Trainingsanwendung zu entwickeln, welche es Spielern ermöglicht Eröffnungen interaktiv zu erlernen. Dabei soll die Anwendung nicht nur die grundlegenden Prinzipien und Variationen der Eröffnungen vermitteln, sondern auch eine personalisierte Lernerfahrung bieten, die sich an den individuellen Fortschritt und die spezifischen Bedürfnisse der Spieler anpasst.

\section{Geplantes Vorgehen}
Zu Beginn wird untersucht, wie das menschliche Gehirn lernt und wie es dabei unterstützt werden kann. Darauffolgend wird erläutert, wie Schachcomputer funktionieren und wie diese mithilfe von Schnittstellen in ein Programm eingebunden werden können. In diesem Zusammenhang werden auch Schnittstellen im Allgemeinen betrachtet.

In Kapitel 3 wird der aktuelle Markt darauf untersucht, welche Lernmaterialien und Tools es bereits gibt. Es wird auch verglichen, welche Schachengines und Datenbanken sich für das geplante Programm eignen.

Anschließend wird die Architektur der Anwendung konzipiert unter Berücksichtigung der technischen und funktionalen Anforderungen. Auf dieser Grundlage wird ein Prototyp entwickelt, der die zentralen Funktionen der Schach-Trainingsanwendung demonstriert.
