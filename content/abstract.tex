%!TEX root = ../main.tex

\pagestyle{empty}

% override abstract headline
\renewcommand{\abstractname}{Abstract}

\begin{abstract}
Im Rahmen dieser Arbeit wurde untersucht, wie Menschen lernen, mit dem Fokus auf Schacheröffnungen. Dabei stellte sich heraus, dass es besonders wichtig ist, Inhalte zu wiederholen.
Im Optimalfall sollten sie in exponentiell größeren Abständen wiederholt werden.
% Im Optimalfall werden sie in exponentiell größer werdenden Abständen wiederholt.
% Im Optimalfall wiederholt man sie mithilfe von Spaced Repetition in exponentiell größer werdenden Abständen.
Das aktuelle Angebot an Lernmedien, bietet diese Funktionalität aber selten. Daher wurde in dieser Arbeit eine Webseite erstellt, auf der Schachspieler Eröffnungen interaktiv betrachten und üben können. Den Nutzern wird zudem eine Empfehlung gegeben, welche Eröffnungen als Nächstes wiederholt werden soll, basierend auf den genannten wissenschaftlichen Erkenntnissen. Das entstandene Produkt lässt sich gut in Verbindung mit bestehenden Lernmedien verwenden.
\end{abstract}
